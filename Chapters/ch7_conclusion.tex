\chapter{Chapter 7. Conclusions and Future Work}

In summary, a comprehensive analysis of SFR structures in the regions immediately upstream and downstream of the Earth's bow shock was carried out. There were 11949 events identified in the magnetosheath across 1051 hours, with 7689 events being identified from the GS-based algorithm and 4260 from the wavelet analysis. For the solar wind, there were 6669 events identified across 676 hours, with 3476 being attributed to identification via the GS-based algorithm and 3193 via wavelet analysis. \textit{In situ} observations of MMS-1, THM-A, THM-B, THM-C, and THM-E spacecraft were utilized. The extended GS-based analysis was used to identify structures with significant remaining plasma flow aligned with the local magnetic field. Wavelet analysis also performed for 76 time periods in the solar wind and 130 time periods in the magnetosheath, corresponding to 676 hours and 1051 hours, respectively. The average magnetic field, velocity, duration, scale size, etc., of these structures were recorded for comparison. In addition to the general parameters, the GS-based method also yielded a unique set of additional parameters that allowed us to evaluate the distributions of the Wal\'en test slope, magnetic helicity density, magnetic flux, and the orientation of the $z$-axes of the structures.

The distributions of a wide variety of parameters generally follow power laws. The magnetic structures identified by the wavelet analysis were also characterized based on MHD quantities ($\sigma_m$, $\sigma_c$, $\sigma_r$). The different criteria assisted us to distinguish different types of events in terms of dynamic and quasi-static structures in the two regions. This work examined the differences in the properties primarily obtained from the GS-based characterization of the identified SFR structures in the solar wind and magnetosheath. The additional findings of this study are summarized as follows.
\begin{enumerate}
    \item The magnitudes of the Wal\'en slope $|w|$ and cross helicity $\sigma_c$ parameters indicate the effect of the remaining plasma flow of the identified structures. Results from both regions indicate that about one-third of structures possess modest (\textit{e.g.}, $|w|>0.3$) remaining flow relative to the total number of structures identified.
    \item SFR structures are generally compressed downstream of the bow shock in the magnetosheath: the scale sizes are smaller while magnetic field strength increases in the magnetosheath. The distributions of the scale sizes and durations of the events shows a power-law trend with more, shorter (in duration and size) events in the magnetosheath than in the solar wind.
    \item Magnetic helicity density per unit volume is about one order of magnitude larger than the corresponding value in the solar wind, implying an overall decrease in volume for the SFRs in the magnetosheath. In addition, the poloidal flux had greater magnitude in the magnetosheath, which indicates more compact, twisted structures.
    \item A significant rotation in the polar angle of the $z$-axis in the magnetosheath is seen when compared to the angle in the solar wind, while the distributions of the azimuthal angle maintain two broad peaks separated by approximately 180 degrees.
    \item Identified structures in the magnetosheath (broadly defined as flux ropes with vortical flows) are likely elongated in the downstream flow direction.
\end{enumerate}

Since this study only used data from the quasi-perpendicular region of the Earth's bow shock, future work would be to perform the identification algorithms and statistical analysis on observations from MMS and THEMIS in the quasi-parallel region. As a complement to the coordinated analysis, taking additional observation intervals directly upstream and downstream of the bow shock could be used to investigate the SFR property changes more directly instead of using an ensemble approach. There are, however, very limited observation periods in which the THEMIS probes are in the right configuration for this type of measurement. Furthermore, the relation between turbulence and the SFRs in the solar wind and magnetosheath could be explored further by implementing some of the procedures in \cite{Zank2:2021, Adhikari:2022, Zhao:2022}. And lastly, including more \textit{in situ} observations from spacecraft such as the Advanced Composition Explorer (ACE), Wind, and Cluster would be beneficial to the work on understanding the differences seen in SFRs in the solar wind and magnetosheath.
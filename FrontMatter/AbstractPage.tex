% The top of your abstract will fill out automatically once you fill in the required fields on the main.tex file. In this file, you will provide your abstract body. Type your abstract body at the bottom of this page directly below the \doublespacing command.

\chapter{Abstract}
     \begin{center}
        \large
        \singlespacing
        \textbf{\thesistitle}\\
        \vspace{0.5cm}
        \large
        \textbf{\studentname}\\
        \vspace{0.5cm}
        \normalsize
        \ifdefined\thesis
        \textbf{A thesis submitted in partial fulfillment of the requirements \\for the degree of \degree}\\  
        \else
        \ifdefined\dissertation
        \textbf{A dissertation submitted in partial fulfillment of the requirements \\for the degree of \degree}\\
        \else
        \textbf{Please identify this document as either a thesis or dissertation on the main.tex in the section at the top that must be filled out.}\\
    \fi
    \fi
        \vspace{1cm}
        \textbf{\department}
        
        \vspace{0.25cm}

        \ifdefined\jointuni
        \textbf{The University of Alabama in Huntsville and  \jointuni}
        \else
        \textbf{The University of Alabama in Huntsville}
    \fi

        
        \vspace{0.1cm}
        \textbf{\gradmonth\ \gradyear}
        


    \end{center}
\vspace{0.1cm}

%****************************************************
%Enter the body of your abstract below. Remember there is a 150 word limit!
%****************************************************
\doublespacing
This dissertation work identifies and characterizes magnetic structures in the solar wind and magnetosheath across the Earth's bow shock. I investigate the differences between the properties of small-scale flux rope (SFR) structures immediately upstream and downstream of the bow shock by employing two data analysis methods: one based on wavelet transforms and the other based on the Grad-Shafranov (GS) detection and reconstruction techniques. In situ magnetic field and plasma data from the Magnetospheric Multiscale (MMS) and Time History of Events and Macroscale Interactions during Substorms (THEMIS) missions are used to identify these coherent structures. 

Previous studies of SFRs have used the extended Grad-Shafranov GS based techniques for the solar wind plasma; however, this algorithm was not used in other regions of near-Earth space such as the magnetosheath. Furthermore, many studies regarding flux ropes and other magnetic structures focus on events identified during specific periods of time, instead of seeking to use multiple spacecraft for a study coordinated in the immediate regions across the bow shock. I characterize magnetic structures with varying degrees of Alfv\'enicity as they move across boundaries in near-Earth space. As I investigate the difference between the properties of the magnetic structures in different near-Earth regions, our aim is to characterize the types of magnetic structures that are identified in the magnetosheath and solar wind in a systematic manner. I identify thousands of SFR event intervals over a total time period of $\sim$1000 hours in each region, and an inventory of events with high magnetic helicity is provided. I report the parameters associated with the SFRs such as scale size, duration, magnetic flux content, and magnetic helicity density, and a direct comparison of the statistical properties of the structures from these two regions is performed.

\clearpage


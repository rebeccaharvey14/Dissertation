% The top of your abstract will fill out automatically once you fill in the required fields on the main.tex file. In this file, you will provide your abstract body. Type your abstract body at the bottom of this page directly below the \doublespacing command.

\chapter{Abstract}
     \begin{center}
        \large
        \singlespacing
        \textbf{\thesistitle}\\
        \vspace{0.5cm}
        \large
        \textbf{\studentname}\\
        \vspace{0.5cm}
        \normalsize
        \ifdefined\thesis
        \textbf{A thesis submitted in partial fulfillment of the requirements \\for the degree of \degree}\\  
        \else
        \ifdefined\dissertation
        \textbf{A dissertation submitted in partial fulfillment of the requirements \\for the degree of \degree}\\
        \else
        \textbf{Please identify this document as either a thesis or dissertation on the main.tex in the section at the top that must be filled out.}\\
    \fi
    \fi
        \vspace{1cm}
        \textbf{\department}
        
        \vspace{0.25cm}

        \ifdefined\jointuni
        \textbf{The University of Alabama in Huntsville and  \jointuni}
        \else
        \textbf{The University of Alabama in Huntsville}
    \fi

        
        \vspace{0.1cm}
        \textbf{\gradmonth\ \gradyear}
        


    \end{center}
\vspace{0.1cm}

%****************************************************
%Enter the body of your abstract below. Remember there is a 350 word limit!
%****************************************************
\doublespacing

%This work identifies and characterizes magnetic structures in the solar wind and magnetosheath across the Earth's bow shock. I investigate the differences between properties of small-scale flux rope (SFR) structures immediately upstream and downstream of the bow shock by employing two data analysis methods: one based on wavelet transforms and the other based on the Grad-Shafranov detection and reconstruction techniques. In situ magnetic field and plasma data from the MMS and THEMIS missions are used to identify coherent structures in a systematic manner. Thousands of SFR event intervals, with varying degrees of Alfv\'enicity, are identified in each region over a total time period of $\sim$1000 hours. An inventory of events with high magnetic helicity is provided, and I report the associated parameters such as scale size, duration, magnetic flux content, and magnetic helicity density. A direct comparison of the statistical properties of the structures from these two regions is performed.

This work identifies and characterizes magnetic structures in the solar wind and magnetosheath across the Earth's bow shock. I investigate the differences between the properties of small-scale flux rope (SFR) structures immediately upstream and downstream of the bow shock by employing two data analysis methods: one based on wavelet transforms and the other based on the Grad-Shafranov (GS) detection and reconstruction techniques. 676 hours in the solar wind, and 1051 hours in the magnetosheath, of in situ magnetic field and plasma data from the Magnetospheric Multiscale (MMS) and Time History of Events and Macroscale Interactions during Substorms (THEMIS) missions were used to identify these coherent structures.

I investigate the difference between the properties of the magnetic structures in different near-Earth regions. The magnetic structures with varying degrees of Alfv\'enicity are characterized in a systematic manner as they move across boundaries in near-Earth space. I identified thousands of SFR event intervals in each region, and established an inventory of events with high magnetic helicity. I report the parameters associated with the SFRs such as scale size, duration, and magnetic helicity density, and a direct comparison of the statistical properties of the structures from these two regions is performed. The GS-based method is extended to identify structures with significant remaining plasma flow aligned with the local magnetic field, and yielded a unique set of additional parameters that allowed us to evaluate the distributions of the Wal\'en test slope, magnetic flux, and the orientation of the $z$-axes of the structures.

In general, it is found that the distributions of various parameters follow power laws. The SFR structures seem to be compressed in the magnetosheath, as compared with their counterparts in the solar wind. A significant rotation in the $z$-axis defining the orientation of the structures is also seen across the bow shock. The implications for the elongation of the SFRs in the magnetosheath along one spatial dimension are also discussed. 

\clearpage


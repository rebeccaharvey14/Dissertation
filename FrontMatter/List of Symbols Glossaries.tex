% This uses the glossaries package. With this package, you can include multiple types of lists, track page numbers if desired, and define new lists. More information can be found at the following sites: https://mirrors.mit.edu/CTAN/macros/latex/contrib/glossaries/glossariesbegin.pdf and https://mirrors.rit.edu/CTAN/macros/latex/contrib/glossaries/glossaries-user.pdf https://www.overleaf.com/learn/latex/Glossaries


%*****************************************
%Define your list of glossary items below. Remember that the entries that you enter in this file will not automatically appear in the List of Symbols. You also have to reference the symbol in the body of your thesis by using the \gls command. 

%symbols
% \newglossaryentry{deg}{name=$^\circ$, description={Degree}}
% \newglossaryentry{grav}{name={1D}, description={Normal gravity environment}}
% \newglossaryentry{wf}{name={\textit{f}}, description={Wear factor}}
% \newglossaryentry{alp}{name={$\alpha$},description={Alpha}}
% \newglossaryentry{theta}{name={$r_O$}, description={ecosystem respiration at reference temperature $T_a=0{^\circ}$C}}
% \newglossaryentry{te}{name={$\tau_e$}, description={precision of the normal distribution of the likelihood}}
% \newglossaryentry{q10}{name={$Q_{10}$}, description={multiplication factor to respiration with 10$^\circ$C increases in $T_a$}}
% \newglossaryentry{phi}{name={$\phi$}, description={vapour pressure deficit response function}}
% \newglossaryentry{del}{name=$\delta$, description={Transition coefficient constant for the design of linear-phase FIR filters which are used to take up space when testing the list of symbols}}

\newglossaryentry{MMS}{name=MMS, description={Magnetospheric Multiscale mission}}
\newglossaryentry{THEMIS}{name=THEMIS, description={Time History of Events and Macroscale Interactions during Substorms mission}}

\newglossaryentry{RE}{name=$R_E$, description={Radius of the Earth, 6372 km}}
\newglossaryentry{AU}{name=AU, description={Astronomical unit: distance between the Earth and the sun, 1.4959$\times 10^8$ km}}
\newglossaryentry{GSE}{name=GSE, description={Geocentric Solar Ecliptic coordinate system}}

\newglossaryentry{IMF}{name=IMF, description={Interplanetary magnetic field}}
\newglossaryentry{SFR}{name=SFR, description={small-scale flux rope}}

\newglossaryentry{GS}{name=GS, description={Grad-Shafranov}}
\newglossaryentry{magnetic flux function}{name=$A$, description={magnetic flux function}}
\newglossaryentry{transformed magnetic flux function}{name=$A'$, description={modified magnetic flux function}}
\newglossaryentry{transverse pressure}{name=$P_t'(A')$, description={transverse pressure as a function of the modified magnetic flux function}}
\newglossaryentry{Rdiff}{name=$R_{diff}$, description={point-wise difference residue between the two parts}}
\newglossaryentry{Rfit}{name=$R_{fit}$, description={residue of the fitting function $P_t'(A')$}}
\newglossaryentry{walen}{name=$w$, description={Wal\'en test slope}}

\newglossaryentry{axial flux}{name=$\Phi_z$, description={approximate axial magnetic flux}}
\newglossaryentry{poloidal flux}{name=$|A_m|$, description={poloidal magnetic flux per meter}}

\newglossaryentry{sigma_c}{name=$\sigma_c$, description={normalized cross helicity}}
\newglossaryentry{sigma_m}{name=$\sigma_m$, description={normalized magnetic helicity}}
\newglossaryentry{sigma_r}{name=$\sigma_r$, description={normalized residual energy}}
\newglossaryentry{phi}{name={$\phi$}, description={azimuthal angle between the GSE $X$-direction and the projection of the $z$-axis onto the $XY$-plane}}
\newglossaryentry{theta}{name={$\theta$}, description={polar angle between the SFR $z$-axis and the GSE $Z$-direction}}
\newglossaryentry{alpha}{name={$\alpha$},description={ average Alfv\'en Mach number squared}}
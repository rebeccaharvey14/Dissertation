\chapter*{Organization}

\paragraph*{Chapter 1: Introduction}
This chapter presents the background knowledge needed to set up this work in the context of near-Earth space. The Earth's magnetosphere is described, with detail being given to the magnetosheath. The concept of turbulence in interplanetary space is introduced, as well as the coherent structures that turbulence can produces. These coherent structures that identified in the solar wind and magnetosheath are the focus of this work.

\paragraph*{Chapter 2: Data sets and pre-processing}
The time series magnetic field and plasma data sets from the Magnetospheric Multiscale (MMS) and Time History of Events and Macroscale Interactions during Substorms (THEMIS) missions are used in this work to identify these coherent structures. Chapter 2 describes the two missions, their instrumentation for observing magnetic fields and plasma data, and the resolution of the data products. Also described are the observation periods from MMS and THEMIS that are used to select data in the solar wind and magnetosheath.

\paragraph*{Chapter 3: Wavelet analysis as a single-spacecraft method}
This chapter first defines wavelet transforms and their mathematical properties. Then, it connects the theory of the wavelet power spectrum with magnetohydrodynamic quantities that are used to characterize different magnetic structures, using single-spacecraft measurements. The chapter then outlines the criteria to identify and categorize magnetic structures using wavelet transforms of the magnetic field and plasma data of the observation periods. Lastly, a statistical analysis of the physical quantities of events identified with this method is presented.

\paragraph*{Chapter 4: Grad-Shafranov reconstruction and automated identification algorithm}
The Grad-Shafranov chapter derives the Grad-Shafranov (GS) equation, which can be used to reconstruct two-dimensional cross-sections of magnetic structures. The GS equation is then expanded to include structures with a velocity flow remaining in the frame of reference of the moving structure. This chapter outlines the detailed basis of the reconstruction and criteria for identification of such structures. 2D reconstructions of selected events are presented, and an analysis of the physical quantities (including those unique to the GS-based method) is given.

\paragraph*{Chapter 5: Coordinated analysis}
The subsection of observation periods with simultaneous observations in the magnetosheath and solar wind are discussed. These unique observations allow for a more direct comparison of the magnetic structures identified in these two regions. A statistical analysis of structures identified by both wavelet analysis and GS-based reconstruction, using the coordinated observation intervals, is presented.

\paragraph*{Chapter 6: Combined analysis}

\paragraph*{Chapter 7: Conclusions and Future Work}

\paragraph*{Appendices}
The appendices provide supplemental information to Chapters \ref{ch:ch2_data} and \ref{ch:ch4_GS}. Appendix \ref{appendix:observation-periods} lists the observation periods from MMS and THEMIS used in the study. Appendix \ref{appendix:gs-flowchart} displays a flow chart of the GS-based automated reconstruction and identification algorithm as implemented by \cite{Hu:2018} and \cite{Zheng:2018}.

% note: 107 periods were used for detection; however of the total events, 5 events overlapped with mine that I had already run and then there was 1 period with data not available and 3 with time periods not really inside the magnetosheath by visual identification of time series data

\section{Instrumentation}
\begin{itemize}
    \item The Fast Plasma Investigation (FPI) includes four dual electron spectrometers (DES) and four dual ion spectrometers (DIS) per spacecraft for a total of 64 plasma spectrometers across the observatories. Each of the four MMS spacecraft has eight high-speed ion sensors and eight high-speed electron sensors
    \item measure the velocity-space distribution of electrons from 10 eV to 30 keV and ions from 10 eV to 30 keV with time resolution of 30 ms, and 150 ms, respectively.
    \item high resolution measurements – when burst data is available – and lower resolution Fast and Slow Survey measurements that are of equivalent resolution to similar particle spectrometers on previous magnetospheric missions.
    \item measure 3D (energy + 2D angle) electron or ion differential directional flux distributions with very high temporal resolution.
    \item measures differential directional flux of magnetospheric electrons and ions
    \item FPI performs rapid collection of electrons and positive ions in phase space densities within and near sites in the Earth’s magnetosphere
    \item at 32 different energies in 512 different directions with uniform coverage of the sky
    \item 3D phase space density for either species at the spacecraft location must be stitched together from the measurements of eight different spectrometers
\end{itemize}

% MMS FGM
\begin{itemize}
    \item These instruments measure DC magnetic field with a resolution of 10 ms, DC electric field with a resolution of 1 ms, electric plasma waves to 100 kHz
    \item measures magnetic field by periodically saturating the ferromagnetic ring cores which in turn induce currents in the sense wires that are modulated by the strength of the ambient field.
    \item two magnetic ring cores sensing time varying magnetic flux in the cores
\end{itemize}


\section{GS-based method}
% The automated detection algorithm moves through different windowed periods of the data, with each window identifying events of certain duration. We employ windows with a minimum duration from approximately 30 seconds (10$\Delta t$) to a maximum duration of 343 data points, which corresponds to approximately 17 minutes for THEMIS data, and 25 min for MMS data. 
% In order to identify two-dimensional magnetic structures from single-spacecraft data \citep{Paschmann:2008}, the in situ magnetic field and plasma data from a specified window of time are transformed into the co-moving frame, notably the de Hoffman-Teller frame \citep{deHoffman-Teller:1950}.
Through a trial-and-error process to determine the optimal orientation of the $z$-axis, the azimuthal and poloidal angles that define this frame transformation are chosen.
% The azimuthal angle $\Phi$ is the rotational angle around the $z$-axis (in the $xy$-plane), and polar angle $\theta$ is the angle between the $z$-axis and the velocity vector of the spacecraft.

\section{Results}
Figures \ref{fig:histogram-orientation}, \ref{fig:histogram-flux}, and \ref{fig:histogram-helicitydensity} are distributions of parameters obtained from the GS-based algorithm only: orientation angles of the $z$-axis of the magnetic structures, local maximum of the scalar flux function $|A_m| = \textnormal{max}(|A(x,y)|)$, and the approximation to the helicity density per unit length.

\begin{figure}
    \centering
    \includegraphics[width=0.6\textwidth]{Figures/Histograms/histogram_Asplit.png}
    \caption{Histograms for the local maximum magnetic flux $|A_m|$ of events identified via GS analysis. The blue lines represent events in the solar wind, and the red lines represent events in the magnetosheath. The dashed lines show where the poloidal magnetic flux per unit length is less than zero.}
    \label{fig:histogram-Asplit}
\end{figure}



%%%%%%%%%%%%%%%%%%%%%% Reconstructions %%%%%%%%%%%%%%%%%%%%%%
% \begin{figure}
%     \centering
%     \includegraphics[width=0.49\textwidth]{Reconstructions/stacked_walenTest_20191109_20191110.png}
%     \includegraphics[width=0.48\textwidth]{Reconstructions/stacked_20090619_20090621.png}
%     \caption[GS event reconstructions]{Left: GS-based reconstruction of an event 9:53:10-9:54:36 UT on 9 November 2019. Top: 2D cross-section, with $\hat{x}_{GSE}=[0.890, 0.268, 0.369]$, $\hat{y}_{GSE}=[0.142, 0.605, -0.784]$, $\hat{z}_{GSE}=[-0.433, 0.750, 0.500]$. Bottom: Associated time series data for MMS-1 in the magnetosheath during this period. Right: GS-based reconstruction of an event from 22:15:35-22:17:07 UT on 19 June 2009. Top: 2D cross-section, with $\hat{x}_{GSE}=[0.761, -0.103, 0.640]$, $\hat{y}_{GSE}=[0.628, 0.365, -0.688]$, $\hat{z}_{GSE}=[-0.163,0.925,0.342]$. Bottom: Associated time series data for THM-C in the magnetosheath during this period.}
%     \label{fig:reconstructions}
% \end{figure}


\begin{figure}
    \centering
    \includegraphics[width=\textwidth]{Figures/Reconstructions/stacked_walenTest_20220219_20220220.png}
    \caption[GS-based event reconstruction for 19 February 2022]{GS-based reconstruction of an event on 12:52:43-12:58:12 UT on 19 February 2022. Top: 2D cross-section, with $\hat{z}=[0.163,-0.925,-0.342]$. Middle: Wal\'en slope for this event. Bottom: $P_t'$ vs. $A'$ curve for this event.}
    \label{fig:reconstruction-Feb2022}
\end{figure}

\section{Conclusions \& Summary}

We find that in the magnetosheath, the magnetic structures identified seem to be compressed relative to the structures in the solar wind. The distributions of the scale sizes and durations of the events shows a linear trend with more, shorter (in duration and size) events in the magnetosheath than in the solar wind. From the GS-based method, we were able to see that the helicity density and poloidal flux were greater in the magnetosheath; these parameters would indicate more compact, twisted structures. From the $z$-axes, we find that the orientation of the polar angle sees a significant change from the solar wind to the magnetosheath: whereas the peak in the distribution of polar angles in the solar wind is around 60-70 degrees, the magnetosheath distribution separates into two peaks, with an almost 180 degree separation. With a wider distribution of polar angles, the magnetic structures in the magnetosheath seem to be less uniform in nature.

\begin{enumerate}
    \item The GS-based reconstruction-based algorithm identifies events with a broader range of cross helicity values than with wavelet analysis.
    \item Static structures were more dominant in both the solar wind and magnetosheath, but there was a higher percentage of quasi-static structures identified in the magnetosheath than in the solar wind. %~10\%
    \item The coordinated analysis shows a direct comparison of the events identified in the magnetosheath and solar wind from simultaneous observations. These results augment the overall analysis, showing trends that align with the extended list.
    \item The identifying spacecraft typically moves through the structures in the magnetosheath in a much more oblique manner when identified with the GS-based method, than when structures are identified in the solar wind.
    \item Identified structures in the magnetosheath (broadly defined as flux ropes with vortical flows) were likely elongated in the downstream flow direction.
\end{enumerate}